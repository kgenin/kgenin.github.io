%%%%%%%%%%%%%%%%%%%%%%%%%%%%%%%%%%%%%%%%%
% long Professional Curriculum Vitae
% LaTeX Template
% Version 1.1 (9/12/12)
%b
% This template has been downloaded from:
% http://www.latextemplates.com
%
% Original author:
% Rensselaer Polytechnic Institute (http://www.rpi.edu/dept/arc/training/latex/resumes/)
%
% Important note:
% This template requires the res.cls file to be in the same directory as the
% .tex file. The res.cls file provides the resume style used for structuring the
% document.
%
%%%%%%%%%%%%%%%%%%%%%%%%%%%%%%%%%%%%%%%%%

%----------------------------------------------------------------------------------------
%	PACKAGES AND OTHER DOCUMENT CONFIGURATIONS
%----------------------------------------------------------------------------------------

\documentclass[12pt]{res} % Use the res.cls style, the font size can be changed to 11pt or 12pt here

\usepackage{helvet} % Default font is the helvetica postscript font
%\usepackage{newcent} % To change the default font to the new century schoolbook postscript font uncomment this line and comment the one above
\usepackage{hyperref}
\usepackage{wasysym}
%\usepackage[misc]{ifsym}
\usepackage{marvosym}

\newsectionwidth{0pt} % Stops section indenting

\begin{document}

%----------------------------------------------------------------------------------------
%	YOUR NAME AND ADDRESS(ES) SECTION
%----------------------------------------------------------------------------------------

\name{KONSTANTIN GENIN\\\\ }% konstantin.genin@gmail.com \\ konstantingenin.com\\ } % Your name at the top

% If you don't want one of the addresses, simply remove all the text in the first or second \address{} bracket

\address{Cluster of Excellence \\ Machine Learning: New Perspectives for Science \\ Eberhard Karls Universität Tübingen \\ Tübingen, Germany }  % Your address 1}

\address{\Letter \hspace{1pt} konstantin.genin@uni-tuebingen.de \\ \Mobilefone \hspace{1pt} +49 0174 8914209 \\ \Mundus \hspace{1pt} konstantingenin.com \\ }
%\address{{\bf Permanent Address} \\ 5833 Hobart Street \\ Pittsburgh, PA 15217  \\ cell: +1 718.637.1493 }

%----------------------------------------------------------------------------------------

\begin{resume}
%{\bf AOS}: Philosophy of Science, Formal Epistemology, Philosophy of Statistics and Machine Learning\\
%{\bf AOC}: Political Philosophy, Philosophy of Social Science, Rational Choice, Logic

%----------------------------------------------------------------------------------------
%	OBJECTIVE SECTION
%----------------------------------------------------------------------------------------

\section{\centerline{LIST OF PUBLICATIONS}} 

\vspace{8pt} % Gap between title and text

Konstantin Genin, Conor Mayo-Wilson (2020). {\href{https://www.cmu.edu/dietrich/causality/CameraReadys-accepted\%20papers/46\%5cCameraReady\%5c2_LinGAM_Neurips_Camera_Ready.pdf}{``Statistical Decidability in Linear, Non-Gaussian Models,''}} Spotlight in {\em Causal Discovery and Causality-Inspired Machine Learning Workshop} at the {\em Thirty-Fourth Conference on Neural Information Processing Systems (NeurIPS, 2020)}.

Konstantin Genin, Franz Huber (2020). {\href{https://plato.stanford.edu/entries/formal-belief/}{``Formal Representations of Belief,''}} in Edward N. Zalta, ed., {\em The Stanford Encyclopedia of Philosophy.}

Konstantin Genin (2019). {``Full and Partial Belief,''} in Richard Pettigrew and Jonathan Weisberg, eds., \href{https://philpapers.org/archive/PETTOH-2.pdf}{{\em The Open Handbook of Formal Epistemology.}} PhilPapers Foundation. pp. 437-498.   

Konstantin Genin, Kevin T. Kelly (2018). \href{https://doi.org/10.1007/s11225-018-9809-5}{{``Theory Choice, Theory Change and Inductive Truth-Conduciveness,''}} {\em Studia Logica,} 107(5): 948-989. 

Konstantin Genin, Kevin T. Kelly (2017). \href{https://doi.org/10.4204/EPTCS.251.17}{``The Topology of Statistical Verifiability,''} in J\'{e}r\^{o}me Lang, ed., {\em Proceedings of the Sixteenth Conference on Theoretical Aspects of Rationality and Knowledge} (TARK), pp. 236-250. 

Kevin T. Kelly, Konstantin Genin, Hanti Lin (2016). \href{https://doi.org/10.1007/s11229-015-0993-9}{``Realism, Rhetoric, and Reliability,''} {\em Synthese,} 193(4): 1191-1223. 

Konstantin Genin, Kevin T. Kelly (2015). {``Theory Choice, Theory Change, and Inductive Truth-Conduciveness,''} in R. Ramanujam, ed., \href{https://www.imsc.res.in/tark/TARK2015-proceedings.pdf}{{\em Proceedings of the Fifteenth Conference on Theoretical Aspects of Rationality and Knowledge}} (TARK), pp. 111-121.\\ 

Kevin T. Kelly, Konstantin Genin (2014). \href{http://dx.doi.org/10.1057/9781137403865_9}{``Complexity, Ockham's Razor, and Truth,''} in M. Lissack and A. Graber, eds.,  {\em Modes of Explanation: Affordances for Action and Prediction.} Palgrave Macmillian, pp. 121-131.

Ryan Carlson, Konstantin Genin, Martina Rau, Richard Scheines (2013). ``Student Profiling from Tutoring System Log Data: When do Multiple Graphical Representations Matter?'' in S.K. D'Mello et. al. eds., \href{http://www.educationaldatamining.org/EDM2013/proceedings/EDM2013Proceedings.pdf}{\em Proceedings of the 6th International Conference on Educational Data Mining} (EDM, 2013), pp. 12-20.

\end{resume} 
\end{document}