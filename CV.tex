%%%%%%%%%%%%%%%%%%%%%%%%%%%%%%%%%%%%%%%%%
% long Professional Curriculum Vitae
% LaTeX Template
% Version 1.1 (9/12/12)
%
% This template has been downloaded from:
% http://www.latextemplates.com
%
% Original author:
% Rensselaer Polytechnic Institute (http://www.rpi.edu/dept/arc/training/latex/resumes/)
%
% Important note:
% This template requires the res.cls file to be in the same directory as the
% .tex file. The res.cls file provides the resume style used for structuring the
% document.
%
%%%%%%%%%%%%%%%%%%%%%%%%%%%%%%%%%%%%%%%%%

%----------------------------------------------------------------------------------------
%	PACKAGES AND OTHER DOCUMENT CONFIGURATIONS
%----------------------------------------------------------------------------------------

\documentclass[12pt]{res} % Use the res.cls style, the font size can be changed to 11pt or 12pt here

\usepackage{helvet} % Default font is the helvetica postscript font
%\usepackage{newcent} % To change the default font to the new century schoolbook postscript font uncomment this line and comment the one above

\usepackage{wasysym}
%\usepackage[misc]{ifsym}
\usepackage{marvosym}

\newsectionwidth{0pt} % Stops section indenting

\begin{document}

%----------------------------------------------------------------------------------------
%	YOUR NAME AND ADDRESS(ES) SECTION
%----------------------------------------------------------------------------------------

\name{KONSTANTIN GENIN\\\\ }% konstantin.genin@gmail.com \\ konstantingenin.com\\ } % Your name at the top

% If you don't want one of the addresses, simply remove all the text in the first or second \address{} bracket

\address{Department of Philosophy \\ Carnegie Mellon University \\ 5000 Forbes Avenue Pittsburgh, PA 15213}  % Your address 1}

\address{\Letter \hspace{1pt} konstantin.genin@gmail.com \\ \Mobilefone \hspace{1pt} +1 718.637.1493 \\ \Mundus \hspace{1pt} konstantingenin.com \\ }
%\address{{\bf Permanent Address} \\ 5833 Hobart Street \\ Pittsburgh, PA 15217  \\ cell: +1 718.637.1493 }

%----------------------------------------------------------------------------------------

\begin{resume}
%{\bf AOS}: Philosophy of Science, Formal Epistemology, Philosophy of Statistics and Machine Learning\\
%{\bf AOC}: Political Philosophy, Philosophy of Social Science, Rational Choice, Logic

%----------------------------------------------------------------------------------------
%	OBJECTIVE SECTION
%----------------------------------------------------------------------------------------

\section{\centerline{AREA OF SPECIALIZATION}}

\vspace{8pt} % Gap between title and text

Philosophy of Science, Formal Epistemology, Philosophy of Statistics and Machine Learning\\

\section{\centerline{AREA OF TEACHING COMPETENCE}}

\vspace{8pt} % Gap between title and text

Political Philosophy, Philosophy of Social Science, Rational Choice, Logic\\
 

%----------------------------------------------------------------------------------------
%	EDUCATION SECTION
%----------------------------------------------------------------------------------------

\section{\centerline{EDUCATION}} 

\vspace{10pt} % Gap between title and text

{\sl Doctor of Philosophy}, 
Logic, Computation and Methodology \hfill Expected Autumn 2017 \\ 
Department of Philosophy, Carnegie Mellon University\\
Dissertation Title: ``Ockham's Razor and the Topology of Statistical Inquiry.''\\
Dissertation Advisor: Kevin T. Kelly.

{\sl Master of Science}, 
Logic, Computation and Methodology \hfill January 2015 \\ 
Department of Philosophy, Carnegie Mellon University\\  
Thesis Title: ``Theory Choice, Theory Change, and Inductive Truth-Conduciveness.''\\
Thesis Advisor: Kevin T. Kelly.

 
{\sl Bachelors of Arts}, Mathematics and Philosophy \hfill May 2009 \\ 
Departments of Mathematics and Philosophy resp., Brown University\\
Magna Cum Laude

%----------------------------------------------------------------------------------------
 
\vspace{0.2in} % Some whitespace between sections

%----------------------------------------------------------------------------------------
%	PROFESSIONAL EXPERIENCE SECTION
%----------------------------------------------------------------------------------------

\section{\centerline{PUBLICATIONS}} 

\vspace{8pt} % Gap between title and text

Konstantin Genin, Kevin T. Kelly. ``The Topology of Statistical Verifiability.'' Forthcoming in {\em Proceedings of the Sixteenth Conference on Theoretical Aspects of Rationality and Knowledge} (TARK), 2017.


Konstantin Genin, Kevin T. Kelly. ``Learning, Theory Choice, and Belief Revision.'' Forthcoming in {\em Studia Logica}.

Kevin T. Kelly, Konstantin Genin, Hanti Lin ``Realism, Rhetoric, and Reliability.'' {\em Synthese} 193.4:  1191-1223, 2016.

Konstantin Genin, Kevin T. Kelly. ``Theory Choice, Theory Change, and Inductive Truth-Conduciveness.'' {\em Proceedings of the Fifteenth Conference on Theoretical Aspects of Rationality and Knowledge} (TARK), 2015.

Kevin T. Kelly, Konstantin Genin. ``Complexity, Ockham's Razor, and Truth.'' Collected in {\em Modes of Explanation: Affordances for Action and Prediction.} Lissack, Michael ed. Palgrave Macmillian, 2014.

Ryan Carlson, Konstantin Genin, Martina Rau, Richard Scheines. ``Student Profiling from Tutoring System Log Data: When do Multiple Graphical Representations Matter?'' {\em Proceedings of the Educational Data Mining Conference}, 2013.

%----------------------------------------------------------------------------------------

%----------------------------------------------------------------------------------------
 
\vspace{0.2in} % Some whitespace between sections

%----------------------------------------------------------------------------------------
%	PROFESSIONAL EXPERIENCE SECTION
%----------------------------------------------------------------------------------------

\section{\centerline{WORKS IN PROGRESS}} 

\vspace{8pt} % Gap between title and text

Kevin T. Kelly, Konstantin Genin. {\em Simplicity, and Truth: A Topological Vindication of Inductive Inference and Ockham's Razor.} (Book in progress) 
%----------------------------------------------------------------------------------------


\vspace{0.2in} % Some whitespace between sections

%----------------------------------------------------------------------------------------
%	TALKS SECTION
%----------------------------------------------------------------------------------------

\section{\centerline{TALKS}}

\vspace{8pt} % Gap between title and text

Reply to ``Two Cheers for Akrasia'' (Kevin Dorst) \hfill January 2018\\
Meeting of the American Philosophical Association Eastern Division,\\ 
Savannah, Georgia.

``The Topology of Statistical Verifiability'' \hfill July 2017\\
$XVI^{th}$ Conference on Theoretical Aspects of Rationality and Knowledge,\\
University of Liverpool. 

``How Inductive is Bayesian Conditioning?'' \hfill July 2017\\
Workshop in Experience and Updating,\\
University Bochum, Germany.

``The Topology of Statistical Inquiry.'' \hfill June 2017\\
Workshop in Philosophy and Physical Computing, \\
Virginia Tech, Blacksburg (Invited Talk).

``What is Statistical Deduction?'' \hfill June 2017\\
Workshop in Modality and Method, \\
CMU, Pittsburgh.

Reply to ``Credal Omniscience and Relevance Confirmation.'' (Joel Pust) \hfill March 2017\\
Meeting of the American Philosophical Association Central Division,\\ 
Kansas City.

``Deduction, Induction, Statistics and Topology.'' \hfill November 2016\\
with Kevin T. Kelly,\\
 Workshop in the Logical Structure of Correlated Information Change,\\ Institute for Logic, Language and Computation, Amsterdam.

``A Topological Explanation of Empirical Simplicity.'' \hfill November 2016\\
with Kevin T. Kelly,\\
Philosophy of Science Association Meeting, \\
Altanta.

``Deduction, Induction, and Statistical Inference.'' \hfill September 2016\\
with Kevin T. Kelly,\\
Philosophy of Scientific Experimentation 5,\\
University of Belgrade.

``Simplicity and Scientific Questions.'' \hfill June 2016\\ 
Questions and Attitudes Workshop,\\
Carnegie Mellon Univeristy, Pittsburgh.

``Theory Choice, Theory Change, and Inductive Truth Conduciveness.''\\
\begin{enumerate}
\item Bristol-Gr\"{o}ningen Conference in Formal Epistemology, \hfill July 2015 \\University of Bristol. 
\item $XV^{th}$ Conference on Theoretical Aspects of Rationality and Knowledge, \hfill June 2015\\ Carnegie Mellon. 
\item Formal Epistemology Workshop, \hfill May 2015\\ University of Washington, St. Louis. 
\item CSLI Workshop on Logic, Rationality, and Intelligent Interaction, \hfill May 2015\\ Stanford (Invited Talk). 
\end{enumerate}

``A Topological Theory of Empirical Simplicity.'' \hfill November 2014\\
with Kevin T. Kelly, Hanti Lin,\\
Philosophy of Science Association Meeting,\\
Chicago.

``Learning with Ockham: Simplicity in Inductive Inference.'' \hfill October 2014\\
Cool Logic Seminar,\\ 
Institute for Logic, Language and Computation, Amsterdam.

``An Epistemic Justification of Ockham's Razor'' \hfill October 2014\\ with Kevin T. Kelly,\\ Ren\'{e} Descartes Lectures,\\
Tilburg University.

``The St. Petersburg Paradox.'' \hfill July 2014\\
with Remco Heesen,\\
Swiss Institute Exhibition,\\ 
New York City. 


``Contraction and the Loss of True Belief.'' \\
with Ted Shear,\\
\begin{enumerate}
\item North American Summer School in Logic, Language, and Information, \hfill June 2014\\ Univeristy of Maryland, College Park. 
\item Canadian Society for History and Philosophy of Science Meeting, \hfill May 2014\\ St. Catherine's, Ontario. 
\item Association of Symbolic Logic North American Meeting, \hfill May 2014\\ University of Colorado, Boulder. 
\item Colombian Conference in Logic, Epistemology and Phil. of Science, \hfill February 2014\\ Universidad de Los Andes, Bogota. 
\end{enumerate}

``Tracking and Statistical Knowledge.'' \hfill January 2014\\
 11th Annual Graduate Student Conference in Epistemology,\\
University of Miami. 


``When do Multiple Graphical Representations Matter?'' \hfill July 2013 \\
with Ryan Carlson, et. al.\\
 Educational Data Mining Conference,\\ Memphis. 

``Empirical Simplicity, Efficient Inquiry, and Ockham's Razor.'' \hfill June 2013\\
with Kevin T. Kelly, Hanti Lin,\\
 Workshop on the Logic of Simplicity,\\
Carnegie Mellon, Pittsburgh. 



%----------------------------------------------------------------------------------------

\vspace{0.2in} % Some whitespace between sections

%----------------------------------------------------------------------------------------
%	TEACHING SECTION
%----------------------------------------------------------------------------------------

\section{\centerline{TEACHING EXPERIENCE}} 

\vspace{15pt} % Gap between title and text
{\em Course Instructor}, Carnegie Mellon University \\
Introduction to Political Philosophy \hfill Summer 2017\\
Introduction to Philosophy \hfill Fall 2016\\
Causation, Law and Social Policy \hfill Spring 2016\\
Introduction to Philosophy \hfill Summer 2015\\
Introduction to Philosophy \hfill Summer 2014

{\em TA or Grader}, Carnegie Mellon University \\
Social Structure, Public Policy and Ethics \hfill Spring 2017\\
Philosophy of Religion \hfill Spring 2014\\
Philosophy and Psychology \hfill Fall 2013\\
Social Structure, Public Policy and Ethics \hfill  Spring 2013 
%----------------------------------------------------------------------------------------

\vspace{0.2in} % Some whitespace between sections

%----------------------------------------------------------------------------------------
%	SERVICE SECTION
%----------------------------------------------------------------------------------------

\section{\centerline{DISCIPLINARY SERVICE}} 

\vspace{15pt} % Gap between title and text

Referee, Sixth International Conference on Logic, Rationality and Interaction \hfill May 2017\\
Referee, {\em Erkenntnis} \hfill May 2017\\
Organizer, Pitt-CMU Grad Conference in Philosophy \hfill March 2017\\ 
Referee, {\em Episteme} \hfill December 2016\\
Referee, {\em Erkenntnis} \hfill January 2016\\
Referee, {\em Ergo} \hfill June 2015\\
Referee, {\em Erkenntnis} \hfill May 2015\\
Referee, {\em British Journal for Philosophy of Science} \hfill February 2015\\
Referee, {\em Studies in History and Philosophy of Science} \hfill July 2014\\
Referee, {\em analytica} \hfill December 2014\\
Referee, Pitt-CMU Grad Conference in Philosophy \hfill Fall 2014\\
Program Committee, NASSLLI \hfill Summer 2014


%----------------------------------------------------------------------------------------


\vspace{0.2in} % Some whitespace between sections

%----------------------------------------------------------------------------------------
%	MEMBERSHIPS SECTION
%----------------------------------------------------------------------------------------

\section{\centerline{MEMBERSHIPS}} 

\vspace{-5pt} % Reduce space between section title and contents

\begin{center}
American Philosophical Association \\
Philosophy of Science Association
\end{center}

%----------------------------------------------------------------------------------------

\vspace{0.1in} % Some whitespace between sections

%----------------------------------------------------------------------------------------
%	HONORS SECTION
%----------------------------------------------------------------------------------------

\section{\centerline{OTHER WORK EXPERIENCE}} 

\vspace{15pt} % Reduce space between section title and contents

Assistant Economist, Federal Reserve Bank of New York \hfill Nov. 2010 - Jul. 2012\\ Financial Intermediation, Research and Statistics

Software Developer, UBS Bank \hfill Jul. 2009 - Nov. 2010\\
Fixed Income Analytics


%----------------------------------------------------------------------------------------

%\vspace{0.2in} % Some whitespace between sections

%----------------------------------------------------------------------------------------
%	INTERESTS SECTION
%----------------------------------------------------------------------------------------

%\section{\centerline{INTERESTS}} 

%\vspace{-5pt} % Reduce space between section title and contents

%\begin{center}
%\end{center} 

%----------------------------------------------------------------------------------------

%\vspace{1in} % Some whitespace between sections
%\section{\centerline{WRITING SAMPLE ATTACHED BELOW}}
%\begin{center}
%I am attaching a first-authored paper forthcoming in {\em Studia Logica}. The submission is 28 pages and approximately 10,000 words, excluding the technical appendix.
%\end{center}
\end{resume} 
\end{document}