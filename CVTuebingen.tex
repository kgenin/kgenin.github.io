%%%%%%%%%%%%%%%%%%%%%%%%%%%%%%%%%%%%%%%%%
% long Professional Curriculum Vitae
% LaTeX Template
% Version 1.1 (9/12/12)
%
% This template has been downloaded from:
% http://www.latextemplates.com
%
% Original author:
% Rensselaer Polytechnic Institute (http://www.rpi.edu/dept/arc/training/latex/resumes/)
%
% Important note:
% This template requires the res.cls file to be in the same directory as the
% .tex file. The res.cls file provides the resume style used for structuring the
% document.
%
%%%%%%%%%%%%%%%%%%%%%%%%%%%%%%%%%%%%%%%%%

%----------------------------------------------------------------------------------------
%	PACKAGES AND OTHER DOCUMENT CONFIGURATIONS
%----------------------------------------------------------------------------------------

\documentclass[12pt]{res} % Use the res.cls style, the font size can be changed to 11pt or 12pt here

\usepackage{helvet} % Default font is the helvetica postscript font
%\usepackage{newcent} % To change the default font to the new century schoolbook postscript font uncomment this line and comment the one above
\usepackage{hyperref}
%\usepackage{wasysym}
%\usepackage[misc]{ifsym}
\usepackage{marvosym}

\newsectionwidth{0pt} % Stops section indenting

\begin{document}

%----------------------------------------------------------------------------------------
%	YOUR NAME AND ADDRESS(ES) SECTION
%----------------------------------------------------------------------------------------

\name{KONSTANTIN GENIN\\\\ }% konstantin.genin@gmail.com \\ konstantingenin.com\\ } % Your name at the top

% If you don't want one of the addresses, simply remove all the text in the first or second \address{} bracket

\address{Department of Philosophy \\ University of Toronto \\ Toronto, ON }  % Your address 1}

\address{\Letter \hspace{1pt} konstantin.genin@gmail.com \\ \Mobilefone \hspace{1pt} +1 718.637.1493 \\ \Mundus \hspace{1pt} konstantingenin.com \\ }
%\address{{\bf Permanent Address} \\ 5833 Hobart Street \\ Pittsburgh, PA 15217  \\ cell: +1 718.637.1493 }

%----------------------------------------------------------------------------------------

\begin{resume}
%{\bf AOS}: Philosophy of Science, Formal Epistemology, Philosophy of Statistics and Machine Learning\\
%{\bf AOC}: Political Philosophy, Philosophy of Social Science, Rational Choice, Logic

%----------------------------------------------------------------------------------------
%	OBJECTIVE SECTION
%----------------------------------------------------------------------------------------
\vspace{8pt}
\section{\centerline{AREA OF RESEARCH}}

\vspace{10pt} % Gap between title and text

Philosophy of Science, Formal Epistemology, Philosophy of Machine Learning and Statistics\\


%----------------------------------------------------------------------------------------xb
%	EDUCATION SECTION
%----------------------------------------------------------------------------------------

\section{\centerline{ACADEMIC POSITIONS}} 
\vspace{10pt}

{\sl Postdoctoral Fellow},
Department of Philosophy \hfill September 2018-\\
Faculty of Arts and Sciences, University of Toronto.\\

\section{\centerline{EDUCATION}} 

\vspace{10pt} % Gap between title and text

{\sl Doctor of Philosophy}, 
Logic, Computation and Methodology \hfill August 2018 \\ 
Department of Philosophy, Carnegie Mellon University\\
Dissertation Title: {\em The Topology of Statistical Inquiry.}\\
Dissertation Advisor: Kevin T. Kelly.

{\sl Master of Science}, 
Logic, Computation and Methodology \hfill January 2015 \\ 
Department of Philosophy, Carnegie Mellon University\\  
Thesis Title: {\em Theory Choice, Theory Change, and Inductive Truth-Conduciveness.}\\
Thesis Advisor: Kevin T. Kelly.

 
{\sl Bachelors of Arts}, Mathematics and Philosophy \hfill May 2009 \\ 
Departments of Mathematics and Philosophy resp., Brown University\\
Magna Cum Laude\\

%----------------------------------------------------------------------------------------
 
%\vspace{0.2in} % Some whitespace between sections

%----------------------------------------------------------------------------------------
%	PROFESSIONAL EXPERIENCE SECTION
%----------------------------------------------------------------------------------------

\section{\centerline{TALKS (2017-2020)}}
\vspace{8pt} % Gap between title and text

``Simplicity and Scientific Progress'' \hfill February 2020\\
Symposium: ``Epistemology Meets Philosophy of Statistics''\\
Central Division Meeting of the American Philosophical Association\\
Chicago, Illinois. (Invited Talk).


``Topological Learning Theory'' \hfill June 2019\\
Workshop in Philosophy and Physical Computing, \\
Virginia Tech, Blacksburg. (Invited Talk).

\newpage

``Progressive Methods for Statistical Inquiry'' \hfill March 2019\\
Statistics Department Seminar,\\
Washington University, St. Louis. (Invited Talk)

``Inductive vs. Deductive Statistical Inference'' \hfill November 2018\\
26th Biennial Meeting of the Philosophy of Science Association,\\
Seattle, Washington.

``The Topology of Statistical Inquiry'' \hfill October 2018\\
Workshop on Logic, Information, and Topology,
CMU, Pittsburgh.

``Progressive Methods for Causal Discovery'' \hfill September 2018\\
Workshop on Foundations of Causal Discovery,
CMU, Pittsburgh.

``Topological Epistemology of Science'' with Kevin T. Kelly,  \hfill June  2018\\
North American Summer School of Logic, Language and Information (NASSLLI),\\
CMU, Pittsburgh.

``Simplicity and Scientific Progress'' \hfill June 2018\\
7th CSLI Workshop on Logic, Rationality, and Intelligent Interaction,\\
Stanford, California.

Reply to ``Two Cheers for Akrasia'' (Kevin Dorst) \hfill January 2018\\
Meeting of the American Philosophical Association Eastern Division,\\ 
Savannah, Georgia.

``The Topology of Statistical Verifiability'' \hfill July 2017\\
$XVI^{th}$ Conference on Theoretical Aspects of Rationality and Knowledge,\\
University of Liverpool. 

``How Inductive is Bayesian Conditioning?'' \hfill July 2017\\
Workshop in Experience and Updating,\\
University Bochum, Germany.

``The Topology of Statistical Inquiry.'' \hfill June 2017\\
Workshop in Philosophy and Physical Computing, \\
Virginia Tech, Blacksburg (Invited Talk).

``What is Statistical Deduction?'' \hfill June 2017\\
Workshop in Modality and Method, \\
CMU, Pittsburgh.

Reply to ``Credal Omniscience and Relevance Confirmation.'' (Joel Pust) \hfill March 2017\\
Meeting of the American Philosophical Association Central Division,\\ 
Kansas City.

%----------------------------------------------------------------------------------------

%\vspace{0.2in} % Some whitespace between sections

%----------------------------------------------------------------------------------------
%	TEACHING SECTION
%----------------------------------------------------------------------------------------


%\vspace{0.2in} % Some whitespace between sections

%----------------------------------------------------------------------------------------
%	SERVICE SECTION
%----------------------------------------------------------------------------------------

%\section{\centerline{DISCIPLINARY SERVICE}} 

%\vspace{15pt} % Gap between title and text

%Referee, {\em Synthese} \hfill October 2018\\
%Referee, {\em Journal for General Philosophy of Science} \hfill April 2018\\
%Referee, Sixth International Conference on Logic, Rationality and Interaction \hfill May 2017\\
%Referee, {\em Erkenntnis} \hfill May 2017\\
%Organizer, Pitt-CMU Grad Conference in Philosophy \hfill March 2017\\ 
%Referee, {\em Episteme} \hfill December 2016\\
%Referee, {\em Erkenntnis} \hfill January 2016\\
%Referee, {\em Ergo} \hfill June 2015\\
%Referee, {\em Erkenntnis} \hfill May 2015\\
%Referee, {\em British Journal for Philosophy of Science} \hfill February 2015\\
%Referee, {\em Studies in History and Philosophy of Science} \hfill July 2014\\
%Referee, {\em analytica} \hfill December 2014\\
%Referee, Pitt-CMU Grad Conference in Philosophy \hfill Fall 2014\\
%Program Committee, NASSLLI \hfill Summer 2014


%----------------------------------------------------------------------------------------


\vspace{0.2in} % Some whitespace between sections

%----------------------------------------------------------------------------------------
%	MEMBERSHIPS SECTION
%----------------------------------------------------------------------------------------


\end{resume} 
\end{document}