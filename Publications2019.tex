%%%%%%%%%%%%%%%%%%%%%%%%%%%%%%%%%%%%%%%%%
% long Professional Curriculum Vitae
% LaTeX Template
% Version 1.1 (9/12/12)
%
% This template has been downloaded from:
% http://www.latextemplates.com
%
% Original author:
% Rensselaer Polytechnic Institute (http://www.rpi.edu/dept/arc/training/latex/resumes/)
%
% Important note:
% This template requires the res.cls file to be in the same directory as the
% .tex file. The res.cls file provides the resume style used for structuring the
% document.
%
%%%%%%%%%%%%%%%%%%%%%%%%%%%%%%%%%%%%%%%%%

%----------------------------------------------------------------------------------------
%	PACKAGES AND OTHER DOCUMENT CONFIGURATIONS
%----------------------------------------------------------------------------------------

\documentclass[12pt]{res} % Use the res.cls style, the font size can be changed to 11pt or 12pt here

\usepackage{helvet} % Default font is the helvetica postscript font
%\usepackage{newcent} % To change the default font to the new century schoolbook postscript font uncomment this line and comment the one above
\usepackage{hyperref}
%\usepackage{wasysym}
%\usepackage[misc]{ifsym}
\usepackage{marvosym}

\newsectionwidth{0pt} % Stops section indenting

\usepackage[left=0.8in,right=0.8in,bottom=.5in]{geometry}

\begin{document}

%----------------------------------------------------------------------------------------
%	YOUR NAME AND ADDRESS(ES) SECTION
%----------------------------------------------------------------------------------------

\name{KONSTANTIN GENIN\\\\ }% konstantin.genin@gmail.com \\ konstantingenin.com\\ } % Your name at the top

% If you don't want one of the addresses, simply remove all the text in the first or second \address{} bracket

\address{Department of Philosophy \\ University of Toronto \\ Toronto, ON }  % Your address 1}

\address{\Letter \hspace{1pt} konstantin.genin@gmail.com \\ \Mobilefone \hspace{1pt} +1 718.637.1493 \\ \Mundus \hspace{1pt} konstantingenin.com \\ }
%\address{{\bf Permanent Address} \\ 5833 Hobart Street \\ Pittsburgh, PA 15217  \\ cell: +1 718.637.1493 }

%----------------------------------------------------------------------------------------

\begin{resume}
%{\bf AOS}: Philosophy of Science, Formal Epistemology, Philosophy of Statistics and Machine Learning\\
%{\bf AOC}: Political Philosophy, Philosophy of Social Science, Rational Choice, Logic

%----------------------------------------------------------------------------------------
%	OBJECTIVE SECTION
%----------------------------------------------------------------------------------------

%----------------------------------------------------------------------------------------
%	PROFESSIONAL EXPERIENCE SECTION
%----------------------------------------------------------------------------------------
\section{\centerline{PUBLICATIONS}} 

\vspace{8pt} % Gap between title and text


Konstantin Genin, Kevin T. Kelly. {Learning, Theory Choice, and Belief Revision.} {\em Studia Logica,} 2018. doi:\href{https://doi.org/10.1007/s11225-018-9809-5}{10.1007/s11225-018-9809-5} 

Konstantin Genin, Kevin T. Kelly.{ The Topology of Statistical Verifiability.} In J\'{e}r\^{o}me Lang, ed., {\em Proceedings of the Sixteenth Conference on Theoretical Aspects of Rationality and Knowledge} (TARK), pages 236-250, 2017. doi:\href{https://doi.org/10.4204/EPTCS.251.17}{10.4204/EPTCS.251.17}


Kevin T. Kelly, Konstantin Genin, Hanti Lin. {Realism, Rhetoric, and Reliability.} {\em Synthese} 193.4:1191-1223, 2016. doi:\href{https://doi.org/10.1007/s11229-015-0993-9}{10.1007/s11229-015-0993-9}

Konstantin Genin, Kevin T. Kelly. {Theory Choice, Theory Change, and Inductive Truth-Conduciveness.} In R. Ramanujam, ed., {\em Proceedings of the Fifteenth Conference on Theoretical Aspects of Rationality and Knowledge} (TARK), pages 111-121, 2015.\\ URL:\href{https://www.imsc.res.in/tark/TARK2015-proceedings.pdf}{https://www.imsc.res.in/tark/TARK2015-proceedings.pdf}

Kevin T. Kelly, Konstantin Genin.{Complexity, Ockham's Razor, and Truth.} In M. Lissack and A. Graber, ed.,  {\em Modes of Explanation: Affordances for Action and Prediction.} Palgrave Macmillian, 2014. doi:\href{http://dx.doi.org/10.1057/9781137403865_9}{10.1057/9781137403865\_9}

Ryan Carlson, Konstantin Genin, Martina Rau, Richard Scheines. Student Profiling from Tutoring System Log Data: When do Multiple Graphical Representations Matter? In S.K. D'Mello et. al. ed., {\em Proceedings of the 6th International Conference on Educational Data Mining} (EDM), 2013.%\\ URL:\href{http://www.educationaldatamining.org/EDM2013/proceedings/EDM2013Proceedings.pdf}{http://www.educationaldatamining.org/EDM2013/proceedings/EDM2013Proceedings.pdf}


%----------------------------------------------------------------------------------------

%----------------------------------------------------------------------------------------
 
%----------------------------------------------------------------------------------------
%	PROFESSIONAL EXPERIENCE SECTION
%----------------------------------------------------------------------------------------

\section{\centerline{WORKS IN PROGRESS}} 

\vspace{8pt} % Gap between title and text
Konstantin Genin. Full and Partial Belief. In Pettigrew, Richard and Weisberg, Jonathan eds., {\em The Open Handbook of Formal Epistemology}. (Invited article in progress) 

Konstantin Genin. {Simplicity and Scientific Progress.} (Article in progress)

Kevin T. Kelly, Hanti Lin, Konstantin Genin. The Miracle Argument for Scientific Realism: A Learning Theoretic Vindication. (Article in progress)

Kevin T. Kelly, Konstantin Genin. {\em Simplicity, and Truth: A Topological Vindication of Inductive Inference and Ockham's Razor.} (Book in progress) 


%----------------------------------------------------------------------------------------

\end{resume} 
\end{document}