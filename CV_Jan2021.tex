%%%%%%%%%%%%%%%%%%%%%%%%%%%%%%%%%%%%%%%%%
% long Professional Curriculum Vitae
% LaTeX Template
% Version 1.1 (9/12/12)
%
% This template has been downloaded from:
% http://www.latextemplates.com
%
% Original author:
% Rensselaer Polytechnic Institute (http://www.rpi.edu/dept/arc/training/latex/resumes/)
%
% Important note:
% This template requires the res.cls file to be in the same directory as the
% .tex file. The res.cls file provides the resume style used for structuring the
% document.
%
%%%%%%%%%%%%%%%%%%%%%%%%%%%%%%%%%%%%%%%%%

%----------------------------------------------------------------------------------------
%	PACKAGES AND OTHER DOCUMENT CONFIGURATIONS
%----------------------------------------------------------------------------------------

\documentclass[12pt]{res} % Use the res.cls style, the font size can be changed to 11pt or 12pt here

\usepackage{helvet} % Default font is the helvetica postscript font
%\usepackage{newcent} % To change the default font to the new century schoolbook postscript font uncomment this line and comment the one above
\usepackage{hyperref}
\usepackage{wasysym}
%\usepackage[misc]{ifsym}
\usepackage{marvosym}

\newsectionwidth{0pt} % Stops section indenting

\begin{document}

%----------------------------------------------------------------------------------------
%	YOUR NAME AND ADDRESS(ES) SECTION
%----------------------------------------------------------------------------------------

\name{KONSTANTIN GENIN\\\\ }% konstantin.genin@gmail.com \\ konstantingenin.com\\ } % Your name at the top

% If you don't want one of the addresses, simply remove all the text in the first or second \address{} bracket

\address{Cluster of Excellence \\ Machine Learning: New Perspectives for Science \\ Eberhard Karls Universität Tübingen \\ Tübingen, Germany }  % Your address 1}

\address{\Letter \hspace{1pt} konstantin.genin@uni-tuebingen.de \\ \Mobilefone \hspace{1pt} +49 0174 8914209 \\ \Mundus \hspace{1pt} konstantingenin.com \\ }
%\address{{\bf Permanent Address} \\ 5833 Hobart Street \\ Pittsburgh, PA 15217  \\ cell: +1 718.637.1493 }

%----------------------------------------------------------------------------------------

\begin{resume}
%{\bf AOS}: Philosophy of Science, Formal Epistemology, Philosophy of Statistics and Machine Learning\\
%{\bf AOC}: Political Philosophy, Philosophy of Social Science, Rational Choice, Logic

%----------------------------------------------------------------------------------------
%	OBJECTIVE SECTION
%----------------------------------------------------------------------------------------

\section{\centerline{AREA OF SPECIALIZATION}}

\vspace{8pt} % Gap between title and text

Philosophy of Machine Learning and Statistics, Philosophy of Science, Formal Epistemology\\

%\section{\centerline{AREA OF TEACHING COMPETENCE}}

%\vspace{8pt} % Gap between title and text

%Political Philosophy, Philosophy of Social Science,  Rational Choice, Logic \\
 

%----------------------------------------------------------------------------------------xb
%	EDUCATION SECTION
%----------------------------------------------------------------------------------------

\section{\centerline{ACADEMIC POSITIONS}} 
\vspace{10pt}

{\sl Leader of Independent Research Group:}\hfill Spring 2020---Present\\
 ``Epistemology and Ethics of Machine Learning,''\\
at the Cluster of Excellence:\\
``Machine Learning: New Perspectives for Science,''\\
Eberhard Karls Universität, Tübingen.

{\sl Postdoctoral Fellow},
Department of Philosophy \hfill Fall 2018---Spring 2020\\
Faculty of Arts and Sciences, University of Toronto.

\section{\centerline{EDUCATION}} 

\vspace{10pt} % Gap between title and text

{\sl Doctor of Philosophy}, 
Logic, Computation and Methodology \hfill Fall 2012---Spring 2018 \\ 
Department of Philosophy, Carnegie Mellon University\\
Dissertation Title: {\em The Topology of Statistical Inquiry.}\\
Dissertation Advisor: Kevin T. Kelly.

{\sl Master of Science}, 
Logic, Computation and Methodology \hfill Fall 2012---Spring 2015 \\ 
Department of Philosophy, Carnegie Mellon University\\  
Thesis Title: {\em Theory Choice, Theory Change, and Inductive Truth-Conduciveness.}\\
Thesis Advisor: Kevin T. Kelly.

 
{\sl Bachelors of Arts}, Mathematics and Philosophy \hfill Fall 2005---Spring 2009 \\ 
Departments of Mathematics and Philosophy resp., Brown University\\
Magna Cum Laude

%----------------------------------------------------------------------------------------
 
\vspace{0.2in} % Some whitespace between sections

%----------------------------------------------------------------------------------------
%	PROFESSIONAL EXPERIENCE SECTION
%----------------------------------------------------------------------------------------
\section{\centerline{PUBLICATIONS}} 

\vspace{8pt} % Gap between title and text

Konstantin Genin, Thomas Grote (2021) {\href{http://philmed.pitt.edu/philmed/article/view/27}{``Randomized Controlled Trials in Medical AI: A Methodological Critique''} {\em Philosophy of Medicine}, 2(1). }

Konstantin Genin, Conor Mayo-Wilson (2020). {\href{https://www.cmu.edu/dietrich/causality/CameraReadys-accepted\%20papers/46\%5cCameraReady\%5c2_LinGAM_Neurips_Camera_Ready.pdf}{``Statistical Decidability in Linear, Non-Gaussian Models,''}} Spotlight in {\em Causal Discovery and Causality-Inspired Machine Learning Workshop} at the {\em Thirty-Fourth Conference on Neural Information Processing Systems (NeurIPS, 2020)}.

Konstantin Genin, Franz Huber (2020). {\href{https://plato.stanford.edu/entries/formal-belief/}{``Formal Representations of Belief,''}} in Edward N. Zalta, ed., {\em The Stanford Encyclopedia of Philosophy.}

Konstantin Genin (2019). {``Full and Partial Belief,''} in Richard Pettigrew and Jonathan Weisberg, eds., \href{https://philpapers.org/archive/PETTOH-2.pdf}{{\em The Open Handbook of Formal Epistemology.}} PhilPapers Foundation. pp. 437-498.   

Konstantin Genin, Kevin T. Kelly (2018). \href{https://doi.org/10.1007/s11225-018-9809-5}{{``Theory Choice, Theory Change and Inductive Truth-Conduciveness,''}} {\em Studia Logica,} 107(5): 948-989. 

Konstantin Genin, Kevin T. Kelly (2017). \href{https://doi.org/10.4204/EPTCS.251.17}{``The Topology of Statistical Verifiability,''} in J\'{e}r\^{o}me Lang, ed., {\em Proceedings of the Sixteenth Conference on Theoretical Aspects of Rationality and Knowledge} (TARK), pp. 236-250. 

Kevin T. Kelly, Konstantin Genin, Hanti Lin (2016). \href{https://doi.org/10.1007/s11229-015-0993-9}{``Realism, Rhetoric, and Reliability,''} {\em Synthese,} 193(4): 1191-1223. 

Konstantin Genin, Kevin T. Kelly (2015). {``Theory Choice, Theory Change, and Inductive Truth-Conduciveness,''} in R. Ramanujam, ed., \href{https://www.imsc.res.in/tark/TARK2015-proceedings.pdf}{{\em Proceedings of the Fifteenth Conference on Theoretical Aspects of Rationality and Knowledge}} (TARK), pp. 111-121.\\ 

Kevin T. Kelly, Konstantin Genin (2014). \href{http://dx.doi.org/10.1057/9781137403865_9}{``Complexity, Ockham's Razor, and Truth,''} in M. Lissack and A. Graber, eds.,  {\em Modes of Explanation: Affordances for Action and Prediction.} Palgrave Macmillian, pp. 121-131.

Ryan Carlson, Konstantin Genin, Martina Rau, Richard Scheines (2013). ``Student Profiling from Tutoring System Log Data: When do Multiple Graphical Representations Matter?'' in S.K. D'Mello et. al. eds., \href{http://www.educationaldatamining.org/EDM2013/proceedings/EDM2013Proceedings.pdf}{\em Proceedings of the 6th International Conference on Educational Data Mining} (EDM, 2013), pp. 12-20.



%----------------------------------------------------------------------------------------

%----------------------------------------------------------------------------------------
 
\vspace{0.2in} % Some whitespace between sections

%----------------------------------------------------------------------------------------
%	PROFESSIONAL EXPERIENCE SECTION
%----------------------------------------------------------------------------------------

%\section{\centerline{WORKS IN PROGRESS}} 

%\vspace{8pt} % Gap between title and text
%Konstantin Genin. {\em Replication as Diachronic Reliability.} (Article in progress)

%Konstantin Genin. {\em Simplicity and Scientific Progress.} (Article in progress)

%Konstantin Genin. {\em Progressive Methods for Causal Discovery.} (Article in progress)

%Kevin T. Kelly, Konstantin Genin. {\em Simplicity, and Truth: A Topological Vindication of Inductive Inference and Ockham's Razor.} (Book in progress) 
%----------------------------------------------------------------------------------------


%\vspace{0.2in} % Some whitespace between sections

%----------------------------------------------------------------------------------------
%	TALKS SECTION
%----------------------------------------------------------------------------------------

\section{\centerline{TALKS}}

\vspace{8pt} % Gap between title and text

``Statistical Decidability in Linear, Non-Gaussian Causal Models''\hfill December, 2020\\
with Conor Mayo-Wilson,\\
Causal Discovery and Causality-Inspired Machine Learning Workshop\\
34th Conference on Neural Information Processing Systems (NeurIPS 2020)\\
Virtual Conference.


``Morals and Methodology''\hfill December, 2020\\
Seminar Series of the Cluster of Excellence:\\
``Machine Learning: New Perspectives for Science''\\
Eberhard Karls Universität, Tübingen (Virtual).

``Simplicity and Scientific Progress''\\
\begin{enumerate}
\item Logic and Philosophy of Science Research Group Seminar, \hfill October 2019\\ Univeristy of Toronto. 
\item American Philosophical Association, Central Division \hfill February 2020\\ Chicago. 
\item Foundations of Probability Seminar, \hfill November 2020\\ Princeton (Virtual). 
\item Logic and Interactive Rationality Seminar, \hfill December 2020\\ Amsterdam (Virtual). 
\end{enumerate}

``Progressive Methods for Causal Discovery'' \hfill August, 2019\\
16th International Congress\\ Logic, Methodology and Philosophy of Science and Technology (CLMPST)\\
Czech Technical University, Prague.

``Topological Learning Theory'' \hfill June, 2019\\
Workshop in Philosophy and Physical Computing,\\
Virginia Tech, Blacksburg.

``Progressive Methods for Statistical Inquiry'' \hfill March, 2019\\
Statistics Department Seminar,\\
Washington University, St Louis. 

``Inductive vs. Deductive Statistical Inference'' \hfill November, 2018\\
26th Biennial Meeting of the Philosophy of Science Association,\\
Seattle, Washington.

``The Topology of Statistical Inquiry'' \hfill October 20, 2018\\
Workshop on Logic, Information, and Topology,
CMU, Pittsburgh.

``Progressive Methods for Causal Discovery'' \hfill September 22, 2018\\
Workshop on Foundations of Causal Discovery,
CMU, Pittsburgh.

``Topological Epistemology of Science'' \hfill June 23-29, 2018\\
with Kevin T. Kelly,\\
North American Summer School of Logic, Language and Information (NASSLLI),\\
CMU, Pittsburgh.

``Simplicity and Scientific Progress'' \hfill June 2-3, 2018\\
7th CSLI Workshop on Logic, Rationality, and Intelligent Interaction,\\
Stanford, California.

Reply to ``Two Cheers for Akrasia'' (Kevin Dorst) \hfill January 2018\\
Meeting of the American Philosophical Association Eastern Division,\\ 
Savannah, Georgia.

``The Topology of Statistical Verifiability'' \hfill July 2017\\
$XVI^{th}$ Conference on Theoretical Aspects of Rationality and Knowledge,\\
University of Liverpool. 

``How Inductive is Bayesian Conditioning?'' \hfill July 2017\\
Workshop in Experience and Updating,\\
University Bochum, Germany.

``The Topology of Statistical Inquiry.'' \hfill June 2017\\
Workshop in Philosophy and Physical Computing, \\
Virginia Tech, Blacksburg (Invited Talk).

``What is Statistical Deduction?'' \hfill June 2017\\
Workshop in Modality and Method, \\
CMU, Pittsburgh.

Reply to ``Credal Omniscience and Relevance Confirmation.'' (Joel Pust) \hfill March 2017\\
Meeting of the American Philosophical Association Central Division,\\ 
Kansas City.

``Deduction, Induction, Statistics and Topology.'' \hfill November 2016\\
with Kevin T. Kelly,\\
 Workshop in the Logical Structure of Correlated Information Change,\\ Institute for Logic, Language and Computation, Amsterdam.

``A Topological Explanation of Empirical Simplicity.'' \hfill November 2016\\
with Kevin T. Kelly,\\
Philosophy of Science Association Meeting, \\
Altanta.

``Deduction, Induction, and Statistical Inference.'' \hfill September 2016\\
with Kevin T. Kelly,\\
Philosophy of Scientific Experimentation 5,\\
University of Belgrade.

``Simplicity and Scientific Questions.'' \hfill June 2016\\ 
Questions and Attitudes Workshop,\\
Carnegie Mellon Univeristy, Pittsburgh.

``Theory Choice, Theory Change, and Inductive Truth Conduciveness.''\\
\begin{enumerate}
\item Bristol-Gr\"{o}ningen Conference in Formal Epistemology, \hfill July 2015 \\University of Bristol. 
\item $XV^{th}$ Conference on Theoretical Aspects of Rationality and Knowledge, \hfill June 2015\\ Carnegie Mellon. 
\item Formal Epistemology Workshop, \hfill May 2015\\ University of Washington, St. Louis. 
\item CSLI Workshop on Logic, Rationality, and Intelligent Interaction, \hfill May 2015\\ Stanford (Invited Talk). 
\end{enumerate}

``A Topological Theory of Empirical Simplicity.'' \hfill November 2014\\
with Kevin T. Kelly, Hanti Lin,\\
Philosophy of Science Association Meeting,\\
Chicago.

``Learning with Ockham: Simplicity in Inductive Inference.'' \hfill October 2014\\
Cool Logic Seminar,\\ 
Institute for Logic, Language and Computation, Amsterdam.

``An Epistemic Justification of Ockham's Razor'' \hfill October 2014\\ with Kevin T. Kelly,\\ Ren\'{e} Descartes Lectures,\\
Tilburg University.

``The St. Petersburg Paradox.'' \hfill July 2014\\
with Remco Heesen,\\
Swiss Institute Exhibition,\\ 
New York City. 


``Contraction and the Loss of True Belief.'' \\
with Ted Shear,\\
\begin{enumerate}
\item North American Summer School in Logic, Language, and Information, \hfill June 2014\\ Univeristy of Maryland, College Park. 
\item Canadian Society for History and Philosophy of Science Meeting, \hfill May 2014\\ St. Catherine's, Ontario. 
\item Association of Symbolic Logic North American Meeting, \hfill May 2014\\ University of Colorado, Boulder. 
\item Colombian Conference in Logic, Epistemology and Phil. of Science, \hfill February 2014\\ Universidad de Los Andes, Bogota. 
\end{enumerate}

``Tracking and Statistical Knowledge.'' \hfill January 2014\\
 11th Annual Graduate Student Conference in Epistemology,\\
University of Miami. 


``When do Multiple Graphical Representations Matter?'' \hfill July 2013 \\
with Ryan Carlson, et. al.\\
 Educational Data Mining Conference,\\ Memphis. 

``Empirical Simplicity, Efficient Inquiry, and Ockham's Razor.'' \hfill June 2013\\
with Kevin T. Kelly, Hanti Lin,\\
 Workshop on the Logic of Simplicity,\\
Carnegie Mellon, Pittsburgh. 



%----------------------------------------------------------------------------------------

\vspace{0.2in} % Some whitespace between sections

%----------------------------------------------------------------------------------------
%	TEACHING SECTION
%----------------------------------------------------------------------------------------

\section{\centerline{TEACHING EXPERIENCE}} 

\vspace{15pt} % Gap between title and text
{\em Course Instructor}, Carnegie Mellon University \\
Causation, Law and Social Policy \hfill Spring 2018\\
Introduction to Political Philosophy \hfill Summer 2017\\
Introduction to Philosophy \hfill Fall 2016\\
Causation, Law and Social Policy \hfill Spring 2016\\
Introduction to Philosophy \hfill Summer 2015\\
Introduction to Philosophy \hfill Summer 2014

{\em TA or Grader}, Carnegie Mellon University \\
Philosophy of Science \hfill Fall 2017\\
Social Structure, Public Policy and Ethics \hfill Spring 2017\\
Philosophy of Religion \hfill Spring 2014\\
Philosophy and Psychology \hfill Fall 2013\\
Social Structure, Public Policy and Ethics \hfill  Spring 2013 
%----------------------------------------------------------------------------------------

\vspace{0.2in} % Some whitespace between sections

%----------------------------------------------------------------------------------------
%	SERVICE SECTION
%----------------------------------------------------------------------------------------

\section{\centerline{DISCIPLINARY SERVICE}} 

\vspace{15pt} % Gap between title and text

Referee, {\em Ergo} \hfill July 2020\\
Referee, {\em Synthese} \hfill July 2020\\
Referee, {\em Philosophy of Science} \hfill October 2019\\
Referee, {\em Synthese} \hfill October 2018\\
Referee, {\em Journal for General Philosophy of Science} \hfill April 2018\\
Referee, Sixth International Conference on Logic, Rationality and Interaction \hfill May 2017\\
Referee, {\em Erkenntnis} \hfill May 2017\\
Organizer, Pitt-CMU Grad Conference in Philosophy \hfill March 2017\\ 
Referee, {\em Episteme} \hfill December 2016\\
Referee, {\em Erkenntnis} \hfill January 2016\\
Referee, {\em Ergo} \hfill June 2015\\
Referee, {\em Erkenntnis} \hfill May 2015\\
Referee, {\em British Journal for Philosophy of Science} \hfill February 2015\\
Referee, {\em Studies in History and Philosophy of Science} \hfill July 2014\\
Referee, {\em analytica} \hfill December 2014\\
Referee, Pitt-CMU Grad Conference in Philosophy \hfill Fall 2014\\
Program Committee, NASSLLI \hfill Summer 2014


%----------------------------------------------------------------------------------------


\vspace{0.2in} % Some whitespace between sections

%----------------------------------------------------------------------------------------
%	MEMBERSHIPS SECTION
%----------------------------------------------------------------------------------------

\section{\centerline{MEMBERSHIPS}} 

\vspace{-5pt} % Reduce space between section title and contents

\begin{center}
American Philosophical Association \\
Philosophy of Science Association
\end{center}

\vspace{0.2in} % Some whitespace between sections

%----------------------------------------------------------------------------------------
%	LANGUAGES SECTION
%----------------------------------------------------------------------------------------

\section{\centerline{LANGUAGES}} 

\vspace{-5pt} % Reduce space between section title and contents

\begin{center}
English --- Native Speaker\\
Russian --- Fluent\\
French --- Intermediate\\
German --- Beginner
\end{center}


%----------------------------------------------------------------------------------------

\vspace{0.1in} % Some whitespace between sections

%----------------------------------------------------------------------------------------
%	HONORS SECTION
%----------------------------------------------------------------------------------------

%\section{\centerline{OTHER WORK EXPERIENCE}} 

%\vspace{15pt} % Reduce space between section title and contents

%Assistant Economist, Federal Reserve Bank of New York \hfill Nov. 2010 - Jul. 2012\\ Financial Intermediation, Research and Statistics

%Software Developer, UBS Bank \hfill Jul. 2009 - Nov. 2010\\
%Fixed Income Analytics


%----------------------------------------------------------------------------------------

%\vspace{0.2in} % Some whitespace between sections

%----------------------------------------------------------------------------------------
%	INTERESTS SECTION
%----------------------------------------------------------------------------------------

%\section{\centerline{INTERESTS}} 

%\vspace{-5pt} % Reduce space between section title and contents

%\begin{center}
%\end{center} 

%----------------------------------------------------------------------------------------

%\vspace{1in} % Some whitespace between sections
%\section{\centerline{WRITING SAMPLE ATTACHED BELOW}}
%\begin{center}
%I am attaching a first-authored paper forthcoming in {\em Studia Logica}. The submission is 28 pages and approximately 10,000 words, excluding the technical appendix.
%\end{center}
\end{resume} 
\end{document}
